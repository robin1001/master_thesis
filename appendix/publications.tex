% !Mode:: "TeX:UTF-8"

\defaultfont
\BiAppendixChapter{攻读\cxuewei 学位期间发表的论文及其他成果} {Papers published in the period of PH.D. education}
\setlength{\parindent}{0em}
\textbf{一、发表的学术论文}
\begin{publist}
  \item {\bf Xiangzeng Zhou}, Lei Xie, Peng Zhang and Yanning Zhang. Online Object Tracking Based on BLSTM-RNN with Contextual-Sequential Labeling[C]. \textcolor{blue}{IEEE Transactions on Multimedia (TMM)} (under review).(\underline{\bf SCI 2区})
\end{publist}
\vspace{1cm}

\textbf{二、博士期间的获奖情况}
\begin{publist}
 \item ICIP 2014 Top\%10 Paper Award.({\bf 第一作者})
 \item 第10届京港国际博士生论坛Best Paper.({\bf 第一作者})
 \item 获2016年西北工业大学大学计算机学院“图灵之星”奖
 \item 获2015年西北工业大学“理光奖学金”
 \item 获2015年西北工业大学“海泰奖学金”
\end{publist}
% \textbf{二、参与的科研项目及获奖情况}
% \begin{publist}
% \item XX. Towards A Queue-Aware ATM: Monitoring and Managing Queues in Front of ATMs, NCR英国国际合作项目.课题编号:XXXX.
% \item XX. 语音内容分析的关键技术研究, 陕西省自然科学基础研究计划.课题编号:XXXX.
% \item XX. 基于DBN协同建模的中文及跨语种语音结构事件检测研究, 国家自然科学基金.课题编号:XXXX.
% \item XX. 机载语音处理技术,	研究所合作项目.课题编号:XXXX.
% \end{publist}
% \vfill
% \hangafter=1\hangindent=2em\noindent

% \setlength{\parindent}{2em}
