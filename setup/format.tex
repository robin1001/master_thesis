% !Mode:: "TeX:UTF-8"



\theoremstyle{plain}
\theorembodyfont{\song\rmfamily}
\theoremheaderfont{\hei\rmfamily}
\newtheorem{definition}{\hei 定义}[chapter]
\newtheorem{example}{\hei 例}[chapter]
\newtheorem{algo}{\hei 算法}[chapter]
\newtheorem{theorem}{\hei 定理}[chapter]
\newtheorem{axiom}{\hei 公理}[chapter]
\newtheorem{proposition}{\hei 命题}[chapter]
\newtheorem{lemma}{\hei 引理}[chapter]
\newtheorem{corollary}{\hei 推论}[chapter]
\newtheorem{remark}{\hei 注解}[chapter]
\newenvironment{proof}{\noindent{\hei 证明:}}{\hfill $ \square $ \vskip 4mm}
\theoremsymbol{$\square$}
\setlength{\theorempreskipamount}{0pt}
\setlength{\theorempostskipamount}{-2pt}

\allowdisplaybreaks[4]

\newcommand{\CJKcaption}[1]{
  \ifx\CJK@actualBinding \@empty
  \PackageError{CJK}{
    You must be inside of a CJK environment to use \protect\CJKcaption}{}
  \else
  \makeatletter
  \InputIfFileExists{#1.cpx}{}{
    \PackageError{CJK}{
      Can't find #1.cpx}{
      The default captions are used if you continue.}}
  \makeatother
  \fi
  }

\CJKcaption{gb_452}
%!%\CJKtilde
\setlength{\parindent}{2em}

\arraycolsep=1.6pt

\renewcommand\contentsname{\hei 目~~~~录}

\renewcommand\chaptername{\CJKprechaptername~\thechapter~\CJKchaptername}

\setcounter{secnumdepth}{4} \setcounter{tocdepth}{2}


\titleformat{\chapter}{\center\hei\fontsize{16pt}{11pt}\selectfont}{\chaptertitlename}{0.5em}{}[\vspace*{-5.5pt}]
%\titleformat{\chapter}{\center\sanhao}{\chaptertitlename}{0.5em}{}
\titlespacing{\chapter}{0pt}{-5.5mm}{8mm}
\titleformat{\section}{\xiaosan\hei}{\thesection}{0.5em}{}
\titlespacing{\section}{0pt}{4.5mm}{4.5mm}
\titleformat{\subsection}{\sihao\hei}{\thesubsection}{0.5em}{}
\titlespacing{\subsection}{0pt}{4mm}{4mm}
\titleformat{\subsubsection}{\xiaosi\hei}{\thesubsubsection}{0.5em}{}
\titlespacing{\subsubsection}{0pt}{0pt}{0pt}

\titlecontents{chapter}[3.8em]{\hspace{-3.8em}\hei}{\CJKprechaptername~\thecontentslabel~\CJKchaptername~~}{}{\titlerule*[4pt]{.}\contentspage}
%\titlecontents{chapter}[3.8em]{\hspace{-3.8em}\hei}{\thecontentslabel~~}{}{\titlerule*[4pt]{.}\contentspage}
\dottedcontents{section}[40pt]{}{22pt}{0.3pc}
\dottedcontents{subsection}[62pt]{}{32pt}{0.3pc}


% 按工大标准, 缩小目录中各级标题之间的缩进,使它们相隔一个字符距离,也就是12pt
\makeatletter
\renewcommand*\l@chapter{\@dottedtocline{0}{0em}{4em}}%控制英文目录: 细点\@dottedtocline  粗点\@dottedtoclinebold
\renewcommand*\l@section{\@dottedtocline{1}{1.5em}{1.8em}}
\renewcommand*\l@subsection{\@dottedtocline{2}{2.5em}{2.5em}}

% 定义页眉和页脚
\newcommand{\makeheadrule}{
\rule[7pt]{\textwidth}{0.7pt} \\[-23pt]
\rule[0.1pt]{\textwidth}{2.8pt}}
\renewcommand{\headrule}{
    {\if@fancyplain\let\headrulewidth\plainheadrulewidth\fi
     \makeheadrule}}
\pagestyle{fancyplain}

%去掉章节标题中的数字
%%不要注销这一行,否则页眉会变成:“第1章1  绪论”样式
\renewcommand{\chaptermark}[1]{\markboth{\chaptertitlename~\ #1}{}}
\fancyhf{}

%在book文件类别下,\leftmark自动存录各章之章名,\rightmark记录节标题
%% 页眉字号 工大要求 小五
%根据单双面打印设置不同的页眉;

\ifxueweidoctor
  \fancyhead[CO]{\song \xiaowu\leftmark}
  \fancyhead[CE]{\song \xiaowu 西北工业大学\cxueke\cxuewei 学位论文 }%
  \fancyfoot[C,C]{\xiaowu ~\thepage~}
\else
  \fancyhead[CO]{\song \xiaowu\leftmark}
  \fancyhead[CE]{\song \xiaowu 西北工业大学\cxueke\cxuewei 学位论文}%
  \fancyfoot[C,C]{\xiaowu ~\thepage~}
\fi

\renewcommand\frontmatter{\cleardoublepage
  \@mainmatterfalse
  \pagenumbering{Roman}}

% 调整罗列环境的布局
\setitemize{leftmargin=3em,itemsep=0em,partopsep=0em,parsep=0em,topsep=-0em}
\setenumerate{leftmargin=3em,itemsep=0em,partopsep=0em,parsep=0em,topsep=0em}

\newcommand{\citeup}[1]{\textsuperscript{\cite{#1}}}
\newcommand{\ucite}[1]{\textsuperscript{\cite{#1}}}

% 定制浮动图形和表格标题样式
\captionnamefont{\wuhao\song}
\captiontitlefont{\wuhao\song}
\captiondelim{~~}
\captionstyle{\centering}
\renewcommand{\subcapsize}{\wuhao}
\renewcommand{\subcapfont}{\song}
\renewcommand{\small}{\wuhao\song}
\setlength{\abovecaptionskip}{0pt}
\setlength{\belowcaptionskip}{0pt}



% 自定义项目列表标签及格式 \begin{publist} 列表项 \end{publist}
\newcounter{pubctr} %自定义新计数器
\newenvironment{publist}{%%%%%定义新环境
\begin{list}{[\arabic{pubctr}]} %%标签格式
    {
     \usecounter{pubctr}
     \setlength{\leftmargin}{2.5em}     % 左边界 \leftmargin =\itemindent + \labelwidth + \labelsep
     \setlength{\itemindent}{0em}     % 标号缩进量
     \setlength{\labelsep}{1em}       % 标号和列表项之间的距离,默认0.5em
     \setlength{\rightmargin}{0em}    % 右边界
     \setlength{\topsep}{0ex}         % 列表到上下文的垂直距离
     \setlength{\parsep}{0ex}         % 段落间距
     \setlength{\itemsep}{0ex}        % 标签间距
     \setlength{\listparindent}{0pt} % 段落缩进量
    }}
{\end{list}}%%%%%

% 默认字体
\renewcommand\normalsize{
  \@setfontsize\normalsize{12pt}{12pt}
  \setlength\abovedisplayskip{12pt}
  \setlength\abovedisplayshortskip{12pt}
  \setlength\belowdisplayskip{\abovedisplayskip}
  \setlength\belowdisplayshortskip{\abovedisplayshortskip}
  \let\@listi\@listI}

% 设置行距和段落间垂直距离
\def\defaultfont{\renewcommand{\baselinestretch}{1.62}\normalsize\selectfont}
\renewcommand{\CJKglue}{\hskip 0.56pt plus 0.08\baselineskip}
%加大字间距,使每行34个字,若要使得每行33个字,则将0.56pt替换为0.96pt。
\predisplaypenalty=0  %公式之前可以换页,公式出现在页面顶部

% 封面、摘要、版权、致谢格式定义
\def\ctitle#1{\def\@ctitle{#1}}\def\@ctitle{}
\def\cdegree#1{\def\@cdegree{#1}}\def\@cdegree{}
\def\caffil#1{\def\@caffil{#1}}\def\@caffil{}
\def\csubject#1{\def\@csubject{#1}}\def\@csubject{}
\def\cauthor#1{\def\@cauthor{#1}}\def\@cauthor{}
\def\studentnum#1{\def\@studentnum{#1}}\def\@studentnum{}
\def\csupervisor#1{\def\@csupervisor{#1}}\def\@csupervisor{}
\def\cassosupervisor#1{\def\@cassosupervisor{{\hei 副 \hfill 导 \hfill 师} & #1\\}}\def\@cassosupervisor{}
\def\ccosupervisor#1{\def\@ccosupervisor{{\hei 联 \hfill 合\hfill 导 \hfill 师} & #1\\}}\def\@ccosupervisor{}
\def\cdate#1{\def\@cdate{#1}}\def\@cdate{}
\long\def\cabstract#1{\long\def\@cabstract{#1}}\long\def\@cabstract{}
\def\ckeywords#1{\def\@ckeywords{#1}}\def\@ckeywords{}

\def\etitle#1{\def\@etitle{#1}}\def\@etitle{}
\def\edegree#1{\def\@edegree{#1}}\def\@edegree{}
\def\eaffil#1{\def\@eaffil{#1}}\def\@eaffil{}
\def\esubject#1{\def\@esubject{#1}}\def\@esubject{}
\def\eauthor#1{\def\@eauthor{#1}}\def\@eauthor{}
\def\esupervisor#1{\def\@esupervisor{#1}}\def\@esupervisor{}
\def\eassosupervisor#1{\def\@eassosupervisor{\textbf{Associate Supervisor:} & #1\\}}\def\@eassosupervisor{}
\def\ecosupervisor#1{\def\@ecosupervisor{\textbf{Co Supervisor:} & #1\\}}\def\@ecosupervisor{}
\def\edate#1{\def\@edate{#1}}\def\@edate{}
\long\def\eabstract#1{\long\def\@eabstract{#1}}\long\def\@eabstract{}
\long\def\NotationList#1{\long\def\@NotationList{#1}}\long\def\@NotationList{}
\def\ekeywords#1{\def\@ekeywords{#1}}\def\@ekeywords{}
\def\natclassifiedindex#1{\def\@natclassifiedindex{#1}}\def\@natclassifiedindex{}
\def\internatclassifiedindex#1{\def\@internatclassifiedindex{#1}}\def\@internatclassifiedindex{}
\def\statesecrets#1{\def\@statesecrets{#1}}\def\@statesecrets{}

% 定义封面
\def\makecover{
    \begin{titlepage}
    % 封面一
    % -- 右上角表格
      \vspace*{0.73cm}
      \begin{minipage}[b]{0.61\linewidth}
        ~~~
      \end{minipage}
      \begin{minipage}[b]{0.3\linewidth}
        \renewcommand{\arraystretch}{1.03}
        {
          \begin{tabular}{|p{1.6cm}|p{1.98cm}|} \hline
            \wuhao\hei\textbf{学\hfill 校\hfill 代\hfill 码}  &  ~\hfill\wuhao\bf 10699 \hfill~               \\ \hline
            \wuhao\hei\textbf{分\hfill 类\hfill 号}           &  ~\hfill\wuhao\@natclassifiedindex\hfill~  \\ \hline
            \wuhao\hei\textbf{密\hfill 级}                    &  ~\hfill\wuhao\song\@statesecrets\hfill~   \\ \hline
            \wuhao\hei\textbf{学\hfill 号}                    &  ~\hfill\wuhao\@studentnum\hfill~          \\ \hline
          \end{tabular}
        }
      \end{minipage}

      \begin{picture}(1,1)(0, 580)
        \put(60, 292){\line(1,0){360}}
        \put(60, 246){\line(1,0){360}}
        \put(170, 178){\line(1,0){120}}
        \put(130, 88){\line(1,0){250}}
        \put(130, 48){\line(1,0){250}}
        \put(130, 8){\line(1,0){250}}
         %\put(130, 581){\line(1,0){700}}
        % \put(130, 492){\line(1,0){700}}
        % \put(340, 355){\line(1,0){220}}
        % \put(250, 203){\line(1,0){520}}
        % \put(250, 125){\line(1,0){520}}
        % \put(250, 50){\line(1,0){520}}
      \end{picture}

      \begin{center}
        \vspace*{8.4cm}
        \begin{tabular}{p{2.1cm}p{12cm}}
           \yihao\song\bfseries ~\hfill 题目 \hfill~ & \fontsize{22pt}{1.85\baselineskip}\selectfont\hei\@ctitle \\
        \end{tabular}

        \vspace*{1.5cm}
        \begin{tabular}{p{1.3cm}p{3.5cm}}
          \sanhao\kai\bfseries ~\hfill 作者 \hfill~ & ~\hfill\fontsize{16pt}{1.5\baselineskip}\selectfont\song\@cauthor\hfill~ \\
        \end{tabular}

        \vspace*{2.3cm}
        \begin{tabular}{p{3.4cm}p{8.7cm}}
          \xiaosan\fs\bfseries 学\hfill 科、\hfill 专\hfill 业 & ~\hfill\fontsize{16pt}{1.5\baselineskip}\selectfont\song\@csubject\hfill~ \\
        \end{tabular}

        \vspace*{0.7cm}
        \begin{tabular}{p{3.4cm}p{8.7cm}}
          \xiaosan\fs\bfseries 指\hfill 导\hfill 教\hfill 师   & ~\hfill\fontsize{16pt}{1.5\baselineskip}\selectfont\song\@csupervisor\hfill~ \\
        \end{tabular}

        \vspace*{0.7cm}
        \begin{tabular}{p{3.4cm}p{8.7cm}}
          \xiaosan\fs\bfseries 申\hfill 请\hfill 学\hfill 位\hfill 日\hfill 期 & ~\hfill\fontsize{16pt}{1.5\baselineskip}\selectfont\song\@cdate\hfill~ \\
        \end{tabular}
      \end{center}

      %\begin{picture}(1,1)(0, 0)
%        \put(60, 292){\line(1,0){360}}
%        \put(60, 246){\line(1,0){360}}
%        \put(170, 178){\line(1,0){120}}
%        \put(130, 100){\line(1,0){250}}
%        \put(130, 60){\line(1,0){250}}
%        \put(130, 20){\line(1,0){250}}
         %\put(130, 581){\line(1,0){700}}
        % \put(130, 492){\line(1,0){700}}
        % \put(340, 355){\line(1,0){220}}
        % \put(250, 203){\line(1,0){520}}
        % \put(250, 125){\line(1,0){520}}
        % \put(250, 50){\line(1,0){520}}
      %\end{picture}

      % 封二 空白页
      \ifxueweidoctor
        \newpage
        ~~~\vspace{1em}
        \thispagestyle{empty}
      \fi

      %内封
      \newpage
      \thispagestyle{empty}
      \begin{center}
        \vspace*{1.6cm}
        \sanhao\song 西~~北~~工~~业~~大~~学

        \vspace*{0.25cm}
        \yihao\song\xueweishort ~~士~~学~~位~~论~~文

        \vspace*{0.75cm}
        \sihao\song (学位研究生)

        \vspace*{4.9cm}
        \begin{tabular}{p{2.1cm}p{9.5cm}}
          \erhao\song ~\hfill 题目: \hfill~ & \fontsize{22pt}{1.5\baselineskip}\selectfont\song\@ctitle \\
        \end{tabular}

        \vspace*{4.0cm}
        \begin{tabular}{p{2.58cm}p{6.2cm}}
          \sanhao\song 作 \hfill 者: & ~\hfill \fontsize{16pt}{1.5\baselineskip}\selectfont\song\@cauthor \hfill~\\
        \end{tabular}

        \vspace*{0.4cm}
        \begin{tabular}{p{2.58cm}p{6.2cm}}
          \sanhao\song 学科专业: & ~\hfill \fontsize{16pt}{1.5\baselineskip}\selectfont\song\@csubject \hfill~\\
        \end{tabular}

        \vspace*{0.4cm}
        \begin{tabular}{p{2.58cm}p{6.2cm}}
          \sanhao\song 指导教师: & ~\hfill \fontsize{16pt}{1.5\baselineskip}\selectfont\song\@csupervisor \hfill~\\
        \end{tabular}

        \vspace*{1.65cm}
        \sanhao\song\@cdate
      \end{center}
      \begin{picture}(1,1)(0, 0)
        \put(125, 325){\line(1,0){280}}
        \put(125, 290){\line(1,0){280}}
        \put(180, 155){\line(1,0){180}}
        \put(180, 123){\line(1,0){180}}
        \put(180, 93){\line(1,0){180}}
        % \put(255, 650){\line(1,0){545}}
        % \put(255, 580){\line(1,0){545}}
        % \put(360, 315){\line(1,0){355}}
        % \put(360, 250){\line(1,0){355}}
        % \put(360, 185){\line(1,0){355}}
      \end{picture}

%%%%%%增加一空白页
  \ifxueweidoctor
    \newpage
    ~~~\vspace{1em}
    \thispagestyle{empty}
  \fi

    % 英文封面
    \newpage
    \thispagestyle{empty}
    % {
    % \xiaosi\noindent Classif\/ied Index: \@natclassifiedindex \\
    %               U.D.C:  \@internatclassifiedindex }
    \begin{center}
      \vspace*{1.20cm} {\erhao\bf Title:~\sanhao\@etitle}

      \vspace*{2.30cm} {\xiaosan\bf By}

      \vspace*{0.15cm} {\xiaosan\bf \@eauthor}

      \vspace*{0.70cm} {\xiaosan\bf Under the Supervision of Professor}

      \vspace*{0.15cm} {\xiaosan\bf \@esupervisor}

      \vspace*{3.00cm} {\xiaosan A Dissertation Submitted to}

      \vspace*{0.15cm} {\xiaosan Northwestern Polytechnical University}

      \vspace*{0.90cm} {\xiaosan In partial fulfillment of the requirement}

      \vspace*{0.15cm} {\xiaosan For the degree of}

      \vspace*{0.15cm} {\xiaosan Master of \bf \xueke}

      \vspace*{2.90cm} {\xiaosan Xi'an P. R. China}

      \vspace*{0.14cm} {\xiaosan \@edate}
    \end{center}
    \end{titlepage}

%%%%%%增加一空白页
  \ifxueweidoctor
    \newpage
    ~~~\vspace{1em}
    \thispagestyle{empty}
  \fi
%%%%%%%%%%%%%%%%%%%   Abstract and keywords  %%%%%%%%%%%%%%%%%%%%%%%
\clearpage

\BiAppendixChapter{摘\quad 要}{Abstract (In Chinese)}

\setcounter{page}{1}
\song\defaultfont
\@cabstract
\vspace{\baselineskip}

\hangafter=1\hangindent=52.3pt%\noindent
{\hei\xiaosi 关键词:} {\@ckeywords}

%% 增加一空白页
% \ifxueweidoctor
%   \newpage
%   ~~~\vspace{1em}
%   \thispagestyle{empty}
% \fi

%%%%%%%%%%%%%%%%%%%   English Abstract  %%%%%%%%%%%%%%%%%%%%%%%%%%%%%%
\clearpage

\phantomsection
\markboth{Abstract}{Abstract}
\addcontentsline{toc}{chapter}{\xiaosi ABSTRACT}
\addcontentsline{toe}{chapter}{\bfseries \xiaosi Abstract (In English)}  \chapter*{\textbf{Abstract}}
\@eabstract
\vspace{\baselineskip}

\hangafter=1\hangindent=60pt\noindent
{\textbf{Key words:}}  \@ekeywords
}

%%%%%%%%%%%%%%%%%%%%%%%%%%%%%%%%%%%%%%%%%%%%%%%%%%%%%%%%%%%%%%%
% 英文目录格式
\def\@dotsep{0.75}           % 定义英文目录的点间距
\setlength\leftmargini {0pt}
\setlength\leftmarginii {0pt}
\setlength\leftmarginiii {0pt}
\setlength\leftmarginiv {0pt}
\setlength\leftmarginv {0pt}
\setlength\leftmarginvi {0pt}

\def\engcontentsname{\bfseries Contents}
\newcommand\tableofengcontents{
   \pdfbookmark[0]{Contents}{econtent}
     \@restonecolfalse
   \chapter*{\engcontentsname  %chapter*上移一行,避免在toc中出现。
       \@mkboth{%
          \engcontentsname}{\engcontentsname}}
   \@starttoc{toe}%
   \if@restonecol\twocolumn\fi
   }

\urlstyle{same}  %论文中引用的网址的字体默认与正文中字体不一致,这里修正为一致的。

\renewcommand\endtable{\vspace{-4mm}\end@float}

\makeatother

