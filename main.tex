% !Mode:: "TeX:UTF-8"
%%%%%%%%%%%%%%%%%%%%%%%%%%%%%%%%%%%%%%%%%%%%%%%%%%%%%%%%%%%%%%%%%%%%%%%%%
%
%   LaTeX File for Doctor (Master) Thesis of Tsinghua University
%   LaTeX + CJK     清华大学博士\KH{硕士}论文模板
%   Based on Wang Tianshu's Template for XJTU
%   Version: 1.00
%   Last Update: 2003-09-12
%
%%%%%%%%%%%%%%%%%%%%%%%%%%%%%%%%%%%%%%%%%%%%%%%%%%%%%%%%%%%%%%%%%%%%%%%%%
%   Copyright 2002-2003  by  Lei Wang (BaconChina)       (bcpub@sina.com)
%%%%%%%%%%%%%%%%%%%%%%%%%%%%%%%%%%%%%%%%%%%%%%%%%%%%%%%%%%%%%%%%%%%%%%%%%

%%%%%%%%%%%%%%%%%%%%%%%%%%%%%%%%%%%%%%%%%%%%%%%%%%%%%%%%%%%%%%%%%%%%%%%%%
%
%   LaTeX File for phd thesis of xi'an Jiao Tong University
%
%%%%%%%%%%%%%%%%%%%%%%%%%%%%%%%%%%%%%%%%%%%%%%%%%%%%%%%%%%%%%%%%%%%%%%%%%
%   Copyright 2002  by  Wang Tianshu    (tswang@asia.com)
%%%%%%%%%%%%%%%%%%%%%%%%%%%%%%%%%%%%%%%%%%%%%%%%%%%%%%%%%%%%%%%%%%%%%%%%%

%%%%%%%%%%%%%%%%%%%%%%%%%%%%%%%%%%%%%%%%%%%%%%%%%%%%%%%%%%%
%
% Latex 西安交通大学博士论文的模板.
%
% 建议使用miktex2.1最大安装编译此模板
%
%%%%%%%%%%%%%%%%%%%%%%%%%%%%%%%%%%%%%%%%%%%%%%%%%%%%%%%%%%


%draft 选项可以使插入的图形只显示外框,以加快预览速度。
%fleqn 让公式左对齐。
%\documentclass[12pt,a4paper,openany,twoside,draft]{book}
\documentclass[12pt,a4paper,openany,twoside]{book}
%\documentclass[11pt,a4paper,openany,draft]{book}
%\documentclass[11pt,a4paper,fleqn,openany,draft]{book}

%以下是采用dvipdfmx所需设置
%\AtBeginDvi{\special{pdf:tounicode GBK-EUC-UCS2}}
%\usepackage[CJKbookmarks=true,dvipdfm,
%           hyperindex=true,
%           pdfstartview=FitH,
%           bookmarksnumbered=true,
%           bookmarksopen=true,
%           colorlinks=true, %注释掉此项则交叉引用为彩色边框(将colorlinks和pdfborder同时注释掉)
%           pdfborder=001,   %注释掉此项则交叉引用为彩色边框
%           citecolor=blue%
%           ]{hyperref}
%%%%%%%%%%%%%%%%%%%%%%%%%%%%%%%%%%%%%%%%%%%%%%%%%%%%%%%%%%%
%
% 引用的宏包
%
%%%%%%%%%%%%%%%%%%%%%%%%%%%%%%%%%%%%%%%%%%%%%%%%%%%%%%%%%%%

% !Mode:: "TeX:UTF-8"
%%%%%%%%%%%%%%%%%%%%%%%%%%%%%%%%%%%%%%%%%%%%%%%%%%%%%%%%%%%%%%%%%%%%%%%%%
%
%   LaTeX File for Doctor (Master) Thesis of Tsinghua University
%   LaTeX + CJK     清华大学博士(硕士)论文模板
%   Based on Wang Tianshu's Template for XJTU
%	Version: 1.00
%   Last Update: 2003-09-12
%
%%%%%%%%%%%%%%%%%%%%%%%%%%%%%%%%%%%%%%%%%%%%%%%%%%%%%%%%%%%%%%%%%%%%%%%%%
%   Copyright 2002-2003  by  Lei Wang (BaconChina)       (bcpub@sina.com)
%%%%%%%%%%%%%%%%%%%%%%%%%%%%%%%%%%%%%%%%%%%%%%%%%%%%%%%%%%%%%%%%%%%%%%%%%

%%%%%%%%%%%%%%%%%%%%%%%%%%%%%%%%%%%%%%%%%%%%%%%%%%%%%%%%%%%%%%%%%%%%%%%%%
%
%   LaTeX File for phd thesis of xi'an Jiao Tong University
%
%%%%%%%%%%%%%%%%%%%%%%%%%%%%%%%%%%%%%%%%%%%%%%%%%%%%%%%%%%%%%%%%%%%%%%%%%
%   Copyright 2002  by  Wang Tianshu    (tswang@asia.com)
%%%%%%%%%%%%%%%%%%%%%%%%%%%%%%%%%%%%%%%%%%%%%%%%%%%%%%%%%%%%%%%%%%%%%%%%%

%%%%%%%%%%%%%%%%%%%%%%%%%%%%%%%%%%%%%%%%%%%%%%%%%%%%%%%%%%%
%
% 引用的宏包和相应的定义
%
%%%%%%%%%%%%%%%%%%%%%%%%%%%%%%%%%%%%%%%%%%%%%%%%%%%%%%%%%%%

%\usepackage[dvips]{graphicx}
\usepackage{graphicx}
\usepackage{subfigure}
% 支持彩色
\usepackage{color}
% eps图像
\usepackage{epsfig}

%\else
%\usepackage[dvips]{graphicx}
%\usepackage{subfigure}
%\fi

% 首行缩进宏包
\usepackage{indentfirst}

% 版面控制宏包,定义规定的版面尺寸
\usepackage[top=0.8in,
	    bottom=1.6in,
	    left=1.2in,
	    right=1.2in,
            %twosideshift=0 pt,
            %headheight=1.0true cm
            ]{geometry}

% 脚注控制
\usepackage[perpage,symbol]{footmisc}

% AMSLaTeX宏包 用来排出更加漂亮的公式
\usepackage{amsmath}
\usepackage{amssymb}
\usepackage{amsthm}

% 不同于\mathcal or \mathfrak 之类的英文花体字体
%\usepackage{mathrsfs}

% 定理类环境宏包,其中 amsmath 选项用来兼容 AMS LaTeX 的宏包
%\usepackage[amsmath,thmmarks]{ntheorem}

% 因为图形可浮动到当前页的顶部,所以它可能会出现
% 在它所在文本的前面. 要防止这种情况,可使用 flafter
% 宏包
%\usepackage{flafter}

%浮动图形控制宏包
%允许上一个section的浮动图形出现在下一个section的开始部分
%该宏包提供处理浮动对象的 \FloatBarrier 命令,使所有未处
%理的浮动图形立即被处理
\usepackage[below]{placeins}

% 图文混排用宏包
%\usepackage{floatflt}

% 图形和表格的控制
\usepackage{rotating}

% tex1cm宏包,控制字体的大小
\usepackage{type1cm}

% 控制标题的宏包
\usepackage[sf]{titlesec}

% 控制目录的宏包
\usepackage{titletoc}

% 处理数学公式中的黑斜体的宏包
\usepackage{bm}

%可将浮动对象放置到文件的最后
%\usepackage{endfloat}

% fancyhdr宏包 页眉和页脚的相关定义
\usepackage{fancyhdr}
\usepackage{fancyref}

% 支持引用的宏包
\usepackage{cite}

%浮动图形和表格标题样式
\usepackage{caption2}

% 定制表格和图形的多行标题行距
\usepackage{setspace}

% 打印当前页面格式的宏包
\usepackage{layouts}

% 使用Times字体的宏包
%\usepackage{times}

% qiuying add
% \usepackage{xeCJK}
% xx add
\usepackage[slantfont,boldfont]{xeCJK}
\punctstyle{quanjiao}
\usepackage{tikz}
\usepackage{listings}
\usepackage{natbib}
\usepackage{algorithm}
\usepackage{algorithmic}

% 生成带书签的pdf
\usepackage[%dvipdfmx,
            CJKbookmarks=true,
            bookmarksnumbered=true,
            bookmarksopen=true,
            colorlinks=true,
            pdfborder=001,
            citecolor=black,
            linkcolor=black,
            anchorcolor=black,
            urlcolor=black,
	  pdftitle={硕士学位论文-基于Web服务的文物监测系统设计与实现},
	  pdfauthor={徐相森},
	  pdfsubject={IPv6},
	  pdfkeywords={tinyos,wsn,web service,sensing node,IPv6,6LoWPAN},
	  pdfcreator={XeTeX,XeCJK},
	  pdfproducer={XeTeX},% 这个好像没起作用?
            ]{hyperref}



\begin{document}

%定义所有的eps文件在 figures 子目录下
\graphicspath{{figures/}}

%%%%%%%%%%%%%%%%%%%%%%%%%%%%%%%%%%%%%%%%%%%%%%%%%%%%%%%%%%%
%
%  文本格式定义
%
%%%%%%%%%%%%%%%%%%%%%%%%%%%%%%%%%%%%%%%%%%%%%%%%%%%%%%%%%%%

% !Mode:: "TeX:UTF-8"
%%%%%%%%%%%%%%%%%%%%%%%%%%%%%%%%%%%%%%%%%%%%%%%%%%%%%%%%%%%%%%%%%%%%%%%%%
%
%   LaTeX File for Doctor (Master) Thesis of Tsinghua University
%   LaTeX + CJK     清华大学博士(硕士)论文模板
%   Based on Wang Tianshu's Template for XJTU
%   Version: 1.00
%   Last Update: 2003-09-12
%
%%%%%%%%%%%%%%%%%%%%%%%%%%%%%%%%%%%%%%%%%%%%%%%%%%%%%%%%%%%%%%%%%%%%%%%%%
%   Copyright 2002-2003  by  Lei Wang (BaconChina)       (bcpub@sina.com)
%%%%%%%%%%%%%%%%%%%%%%%%%%%%%%%%%%%%%%%%%%%%%%%%%%%%%%%%%%%%%%%%%%%%%%%%%

%%%%%%%%%%%%%%%%%%%%%%%%%%%%%%%%%%%%%%%%%%%%%%%%%%%%%%%%%%%%%%%%%%%%%%%%%
%
%   LaTeX File for phd thesis of xi'an Jiao Tong University
%
%%%%%%%%%%%%%%%%%%%%%%%%%%%%%%%%%%%%%%%%%%%%%%%%%%%%%%%%%%%%%%%%%%%%%%%%%
%   Copyright 2002  by  Wang Tianshu    (tswang@asia.com)
%%%%%%%%%%%%%%%%%%%%%%%%%%%%%%%%%%%%%%%%%%%%%%%%%%%%%%%%%%%%%%%%%%%%%%%%%
%%%%%%%%%%%%%%%%%%%%%%%%%%%%%%%%%%%%%%%%%%%%%%%%%%%%%%%%%%%
%
% 主文档 格式定义
%
%%%%%%%%%%%%%%%%%%%%%%%%%%%%%%%%%%%%%%%%%%%%%%%%%%%%%%%%%%%

% 按清华标准, 将版芯控制在240mm以内, 正文范围控制在220mm以内
%\addtolength{\headsep}{-0.1cm}          %页眉位置
%\addtolength{\footskip}{-0.1cm}         %页脚位置
\addtolength{\topmargin}{0.5cm}

%%%%%%%%%%%%%%%%%%%%%%%%%%%%%%%%%%%%%%%%%%%%%%%%%%%%%%%%%%%
% 公式的精调
%%%%%%%%%%%%%%%%%%%%%%%%%%%%%%%%%%%%%%%%%%%%%%%%%%%%%%%%%%%

%\setlength{\mathindent}{4.7 em}     %左对齐公式缩进量

% \eqnarray如果很长,影响分栏、换行和分页(整块挪动,造成页面空白),
% 可以设置成为自动调整模式
\allowdisplaybreaks[4]

%%%%%%%%%%%%%%%%%%%%%%%%%%%%%%%%%%%%%%%%%%%%%%%%%%%%%%%%%%%
%下面这组命令使浮动对象的缺省值稍微宽松一点,从而防止幅度
%对象占据过多的文本页面,也可以防止在很大空白的浮动页上放置
%很小的图形。
%%%%%%%%%%%%%%%%%%%%%%%%%%%%%%%%%%%%%%%%%%%%%%%%%%%%%%%%%%%
\renewcommand{\textfraction}{0.15}
\renewcommand{\topfraction}{0.85}
\renewcommand{\bottomfraction}{0.65}
\renewcommand{\floatpagefraction}{0.60}


%%%%%%%%%%%%%%%%%%%%%%%%%%%%%%%%%%%%%%%%%%%%%%%%%%%%%%%%%%%
%下面这组命令可以使公式编号随着每开始新的一节而重新开始。
%%%%%%%%%%%%%%%%%%%%%%%%%%%%%%%%%%%%%%%%%%%%%%%%%%%%%%%%%%%

%\makeatletter      % '@' is now a normail "letter" for TeX
%\@addtoreset{eqation}{section}
%\makeatother       % '@' is restored as a "non-letter" character for TeX

%%%%%%%%%%%%%%%%%%%%%%%%%%%%%%%%%%%%%%%%%%%%%%%%%%%%%%%%%%%
% 重定义字体命令
%%%%%%%%%%%%%%%%%%%%%%%%%%%%%%%%%%%%%%%%%%%%%%%%%%%%%%%%%%%
% 注意win2000,没有 simsun, 最好到网上找一个
% 一些字体是office2000 带的
%%%%%%%%%%%%%%%%%%%%%%%%%%%%%%%%%%%%%%%%%%%%%%%%%%%%%%%%%%%

\setmainfont{TeX Gyre Termes}
%\setmainfont{Latin Modern Roman}
\setsansfont{TeX Gyre Heros}
\setmonofont{Latin Modern Mono}
%\setCJKmainfont[BoldFont={方正小标宋简体}]{方正书宋简体}    % 宋体
%\setCJKsansfont{Adobe Heiti Std}
%\setCJKmonofont{Adobe Fangsong Std}

%\setCJKfamilyfont{song}[BoldFont={方正宋黑简体}]{SimSun}      	% 宋体
%\setCJKfamilyfont{song}[BoldFont={方正宋三_GBK}]{方正博雅宋_GBK}  % 宋体
%\setCJKfamilyfont{song}[BoldFont={Adobe Heiti Std}]{Adobe Song Std}    % 宋体
%\setCJKfamilyfont{song}[BoldFont={华文中宋}]{华文宋体}    % 宋体
%\setCJKfamilyfont{song}[BoldFont={方正大标宋_GBK}]{方正兰亭宋_GBK}    % 宋体
%\setCJKfamilyfont{song}[BoldFont={方正小标宋简体}]{方正书宋简体}    % 宋体
%\setCJKfamilyfont{hei}{Adobe Heiti Std}      	% 黑体
%\setCJKfamilyfont{hei}{方正兰亭黑_GBK}      	% 黑体
%\setCJKfamilyfont{hei}{黑体}		      	% 黑体
%\setCJKfamilyfont{kai}{Adobe Kaiti Std}      	% 楷体
%\setCJKfamilyfont{fang}{Adobe Fangsong Std}  	% 仿宋体
%\setCJKfamilyfont{nwpulogo}{nwpulogo}        	% 含"西北工业大学"logo字体

% xx comment
\setCJKfamilyfont{song}{SimSun}    	% 宋体
\setCJKfamilyfont{hei}{SimHei}     	% 黑体
\setCJKfamilyfont{kai}{KaiTi}      	% 楷体
\setCJKfamilyfont{fang}{FangSong}  	% 仿宋体

\newcommand{\song}{\CJKfamily{song}}
\newcommand{\hei}{\CJKfamily{hei}}
\newcommand{\fang}{\CJKfamily{fang}}
\newcommand{\kai}{\CJKfamily{kai}}
%\newcommand{\nwpulogo}{\CJKfamily{nwpulogo}}

%%%%%%%%%%%%%%%%%%%%%%%%%%%%%%%%%%%%%%%%%%%%%%%%%%%%%%%%%%%
% 重定义字号命令
%%%%%%%%%%%%%%%%%%%%%%%%%%%%%%%%%%%%%%%%%%%%%%%%%%%%%%%%%%%

\newcommand{\chuhao}{\fontsize{42pt}{63pt}\selectfont}    % 初号, 1.5倍行距
\newcommand{\yihao}{\fontsize{26pt}{36pt}\selectfont}    % 一号, 1.4倍行距
\newcommand{\erhao}{\fontsize{22pt}{28pt}\selectfont}    % 二号, 1.25倍行距
\newcommand{\xiaoer}{\fontsize{18pt}{18pt}\selectfont}    % 小二, 单倍行距
\newcommand{\sanhao}{\fontsize{16pt}{24pt}\selectfont}    % 三号, 1.5倍行距
\newcommand{\xiaosan}{\fontsize{15pt}{22pt}\selectfont}    % 小三, 1.5倍行距
\newcommand{\sihao}{\fontsize{14pt}{21pt}\selectfont}    % 四号, 1.5倍行距
\newcommand{\banxiaosi}{\fontsize{13pt}{16.25pt}\selectfont}    % 半小四, 1.25倍行距
\newcommand{\xiaosi}{\fontsize{12pt}{12pt}\selectfont}    % 小四, 1.2倍行距
\newcommand{\dawuhao}{\fontsize{11pt}{11pt}\selectfont}    % 大五号, 单倍行距
\newcommand{\wuhao}{\fontsize{10.5pt}{10.5pt}\selectfont}    % 五号, 单倍行距
\newcommand{\xiaowu}{\fontsize{9pt}{9pt}\selectfont}		% 小五号



%%%%%%%%%%%%%%%%%%%%%%%%%%%%%%%%%%%%%%%%%%%%%%%%%%%%%%%%%%%
% 重定义一些正文相关标题
%%%%%%%%%%%%%%%%%%%%%%%%%%%%%%%%%%%%%%%%%%%%%%%%%%%%%%%%%%%

% qiuying comment
%\theoremstyle{plain} \theorembodyfont{\song\rmfamily}
%\theoremheaderfont{\hei\rmfamily} \theoremseparator{:}
%\newtheorem{definition}{\hei 定义}[chapter]
%\newtheorem{proposition}[definition]{\hei 命题}
%\newtheorem{lemma}[definition]{\hei 引理}
%\newtheorem{theorem}{\hei 定理}[chapter]
%\newtheorem{axiom}{\hei 公理}
%\newtheorem{corollary}[definition]{\hei 推论}
%\newtheorem{exercise}[definition]{}
%
%\theoremheaderfont{\CJKfamily{hei}\rmfamily}\theorembodyfont{\rmfamily}
%\theoremstyle{nonumberplain} \theoremseparator{:}
%\theoremsymbol{$\blacksquare$}
%\newtheorem{proof}{\hei 证明}
%
%\theoremsymbol{$\square$}
%\newtheorem{example}{\hei 例}
%

%%%%%%%%%%%%%%%%%%%%%%%%%%%%%%%%%%%%%%%%%%%%%%%%%%%%%%%%%%%
% 用于中文段落缩进 和正文版式
%%%%%%%%%%%%%%%%%%%%%%%%%%%%%%%%%%%%%%%%%%%%%%%%%%%%%%%%%%%
%\CJKcaption{GB_aloft}
%\xeCJKcaption{gb_452}

\newlength \CJKtwospaces

\def\CJKindent{
    \settowidth\CJKtwospaces{\CJKchar{"0A1}{"0A1}\CJKchar{"0A1}{"0A1}}%
    \parindent\CJKtwospaces
}


%\CJKtilde  \CJKindent

\renewcommand\contentsname{目~~~~录}
\renewcommand\chaptername{\thechapter}

%%%%%%%%%%%%%%%%%%%%%%%%%%%%%%%%%%%%%%%%%%%%%%%%%%
%定义段落章节的标题和目录项的格式
%%%%%%%%%%%%%%%%%%%%%%%%%%%%%%%%%%%%%%%%%%%%%%%%%%
\setcounter{secnumdepth}{4}
\setcounter{tocdepth}{2}

% Modified By Lei Wang BaconChina
% THU Version
\titleformat{\chapter}[hang]
    {\normalfont\sanhao\filcenter\hei\sf}
    {\sanhao{\chaptertitlename}}
    {20pt}{\sanhao}
%\titlespacing{\chapter}{0pt}{-3ex  plus .1ex minus .2ex}{2.5ex plus .1ex minus .2ex}
\titlespacing{\chapter}{0pt}{-3ex  plus .1ex minus .2ex}{0.25em}

\titleformat{\section}[hang]{\hei \sf \sihao}
    {\sihao \thesection}{0.5em}{}{}
%\titlespacing{\section}{0pt}{1.5ex plus .1ex minus .2ex}{\wordsep}
\titlespacing{\section}{0pt}{0.5em}{0em}

\titleformat{\subsection}[hang]{\hei \sf \banxiaosi}
    {\banxiaosi \thesubsection}{0.5em}{}{}
%    {\banxiaosi \thesubsection}{0pt}{}{}
%\titlespacing{\subsection}{0pt}{1.5ex plus .1ex minus .2ex}{\wordsep}
\titlespacing{\subsection}{0pt}{0.25em}{0em}

\titleformat{\subsubsection}[hang]{\hei \sf}
    {\thesubsubsection }{0.5em}{}{}
%\titlespacing{\subsubsection}{0pt}{1.2ex plus .1ex minus .2ex}{\wordsep}
\titlespacing{\subsubsection}{0pt}{0.25em}{0pt}

%去掉中间对齐的sectionformat,这样就把节的标题左对齐了。
%\renewcommand \sectionformat{}

% 按清华标准, 缩小目录中各级标题之间的缩进
\dottedcontents{chapter}[0.0em]{\vspace{0.5em}}{1.0em}{5pt}
\dottedcontents{section}[0.8cm]{}{1.8em}{5pt}
\dottedcontents{subsection}[1.50cm]{}{2.7em}{5pt}
\dottedcontents{subsubsection}[2.86cm]{}{3.4em}{5pt}

%%%%%%%%%%%%%%%%%%%%%%%%%%%%%%%%%%%%%%%%%%%%%%%%%%%%%%%
% 定义页眉和页脚 使用fancyhdr 宏包
%%%%%%%%%%%%%%%%%%%%%%%%%%%%%%%%%%%%%%%%%%%%%%%%%%%%%%%%

\newcommand{\makeheadrule}{%
    \makebox[0pt][l]{\rule[.73\baselineskip]{\headwidth}{0.5pt}}%
% 1 Line Modified by Lei Wang BaconChina
% XJTU Version
    \rule[.85\baselineskip]{\headwidth}{2.5pt}\vskip-.8\baselineskip}
% THU Version
%    \vskip-.8\baselineskip}



\makeatletter
\renewcommand{\headrule}{%
    {\if@fancyplain\let\headrulewidth\plainheadrulewidth\fi
     \makeheadrule}}

\pagestyle{fancyplain}

%去掉章节标题中的数字
\renewcommand{\chaptermark}[1]{\markboth{\chaptername \ #1}{}}

 \fancyhf{}
% \fancyfoot[C,C]{\thepage}

%在book文件类别下,\leftmark自动存录各章之章名,\rightmark记录节标题

% Modified by Lei Wang BaconChina
% XJTU Version
% \fancyhead[RO]{\CJKfamily{song}\leftmark}
% \fancyhead[LE]{\CJKfamily{song}西安交通大学博士学位论文}
% \fancyfoot[C,C]{--~\thepage~--}
% THU Version
%\fancyhead[CO]{\CJKfamily{song}\wuhao\leftmark}
%\fancyhead[CE]{\nwpulogo\fontsize{8pt}{6pt} 西北工业大学~~~ \sanhao\song 本科毕业设计论文}
%\chead{\song 西北工业大学硕士研究生学位论文}
\fancyhead[CO]{\xiaowu\textbf \leftmark}
\fancyhead[CE]{\xiaowu\textbf{西北工业大学硕士学位论文}}
\fancyfoot[C,C]{\xiaowu\thepage}

%%%%%%%%%%%%%%%%%%%%%%%%%%%%%%%%%%%%%%%%%%%%%%%%%%%%%%%%
% 设置行距和段落间垂直距离
%%%%%%%%%%%%%%%%%%%%%%%%%%%%%%%%%%%%%%%%%%%%%%%%%%%%%%%%

% 段落之间的竖直距离
\setlength{\parskip}{3pt plus1pt minus1pt}

% 定义行距
\renewcommand{\baselinestretch}{1.25}

%%%%%%%%%%%%%%%%%%%%%%%%%%%%%%%%%%%%%%%%%%%%%%%%%%%%%%%%
% 调整列表环境的垂直间距
%%%%%%%%%%%%%%%%%%%%%%%%%%%%%%%%%%%%%%%%%%%%%%%%%%%%%%%%
\let\orig@Itemize =\itemize
\let\orig@Enumerate =\enumerate
\let\orig@Description =\description

\def\Myspacing{\itemsep=5pt \topsep=0pt \partopsep=0pt \parskip=0pt \parsep=0pt}

\def\newitemsep{
\renewenvironment{itemize}{\orig@Itemize\Myspacing}{\endlist}
\renewenvironment{enumerate}{\orig@Enumerate\Myspacing}{\endlist}
\renewenvironment{description}{\orig@Description\Myspacing}{\endlist}
}

\def\olditemsep{
\renewenvironment{itemize}{\orig@Itemize}{\endlist}
\renewenvironment{enumerate}{\orig@Enumerate}{\endlist}
\renewenvironment{description}{\orig@Description}{\endlist}
}

\newitemsep

%%%%%%%%%%%%%%%%%%%%%%%%%%%%%%%%%%%%%%%%%%%%%%%%%%%%%%%
% 修改引用的格式,
%%%%%%%%%%%%%%%%%%%%%%%%%%%%%%%%%%%%%%%%%%%%%%%%%%%%%%%

%第一行在引用处数字两边加方框
%第二行去除参考文献里数字两边的方框
%\makeatletter
%\def\@cite#1{\mbox{$\m@th^{\hbox{\@ove@rcfont[#1]}}$}}
%\renewcommand\@biblabel[1]{#1}
%\makeatother

% 增加 \ucite 命令使显示的引用为上标形式
\renewcommand\bibname{参考文献}

\newcommand{\ucite}[1]{$^{\mbox{\scriptsize \cite{#1}}}$}

%%%%%%%%%%%%%%%%%%%%%%%%%%%%%%%%%%%%%%%%%%%%%%%%%%%%%%%%%%%
%
% 定制浮动图形和表格标题样式
%
%%%%%%%%%%%%%%%%%%%%%%%%%%%%%%%%%%%%%%%%%%%%%%%%%%%%%%%%%%%

% figure 1.1 -> 图 1.1
\renewcommand{\figurename}{图}
% 图 1.1 -> 图 1-1
\renewcommand\thefigure{\arabic{chapter}-\arabic{figure}}
\renewcommand{\captionfont}{\CJKfamily{song}\rmfamily}
\renewcommand{\captionlabelfont}{\CJKfamily{song}\rmfamily}

% xx comment: subfigure capture
%\renewcommand{\thesubfigure}{\alph{subfigure}}
\renewcommand{\p@subfigure}{~\arabic{chapter}-\arabic{figure}~}
\renewcommand{\subcapsize}{\wuhao}

% 按清华标准, 去掉图表号后面的:
\renewcommand{\captionlabeldelim}{\hspace{0.5em}}

% 按清华标准, 图表标题字体为11pt, 这里写作大五号
\renewcommand{\captionfont}{\wuhao}

%%%%%%%%%%%%%%%%%%%%%%%%%%%%%%%%%%%%%%%%%%%%%%%%%%%%%%%
% 定义题头格言的格式
%%%%%%%%%%%%%%%%%%%%%%%%%%%%%%%%%%%%%%%%%%%%%%%%%%%%%%%

%
% 用法 \begin{Aphorism}{author}
%         aphorism
%      \end{Aphorism}

\newsavebox{\AphorismAuthor}
\newenvironment{Aphorism}[1]
{\vspace{0.5cm}\begin{sloppypar} \slshape
\sbox{\AphorismAuthor}{#1}
\begin{quote}\small\itshape }
{\\ \hspace*{\fill}------\hspace{0.2cm} \usebox{\AphorismAuthor}
\end{quote}
\end{sloppypar}\vspace{0.5cm}}

%自定义一个空命令,用于注释掉文本中不需要的部分。
\newcommand{\comment}[1]{}

% This is the flag for longer version
\newcommand{\longer}[2]{#1}

\newcommand{\ds}{\displaystyle}

% define graph scale
\def\gs{1.0}

%%%%%%%%%%%%%%%%%%%%%%%%%%%%%%%%%%%%%%%%%%%%%%%%%%%%%%%%%%%%%%%%%%%%%%
% 自定义项目列表标签及格式 \begin{denselist} 列表项 \end{denselist}
%%%%%%%%%%%%%%%%%%%%%%%%%%%%%%%%%%%%%%%%%%%%%%%%%%%%%%%%%%%%%%%%%%%%%%
\newcounter{newlist} %自定义新计数器
\newenvironment{denselist}[1][可改变的列表题目]{%%%%%定义新环境
\begin{list}{\textbf{\hei #1} \arabic{newlist}:} %%标签格式
    {
    \usecounter{newlist}
     \setlength{\labelwidth}{22pt} %标签盒子宽度
     \setlength{\labelsep}{0cm} %标签与列表文本距离
     \setlength{\leftmargin}{0cm} %左右边界
     \setlength{\rightmargin}{0cm}
     \setlength{\parsep}{0ex} %段落间距
     \setlength{\itemsep}{0ex} %标签间距
     \setlength{\itemindent}{44pt} %标签缩进量
     \setlength{\listparindent}{22pt} %段落缩进量
    }}
{\end{list}}%%%%%

%添加一些有用的命令
%Chinese style for the chapter reference. It doesn't work with hyperref
\newcommand{\chref}[1]{\CJKnumber{\ref{#1}}}
%adjust Chinese parenthesis space
\newcommand{\KH}[1]{\!\!(#1)\!\!}
\newcommand\dlmu@underline[2][5cm]{\hskip1pt\underline{\hb@xt@ #1{\hss#2\hss}}\hskip3pt}
\let\coverunderline\dlmu@underline

\setlength{\parindent}{2em}
\renewcommand{\lstlistingname}{\wuhao 源码}

\setlength{\headheight}{24pt}

%\newfontfamily\pagella{TeX Gyre Pagella}
%\newfontfamily\monaco{Monaco}
%\newfontfamily\droidmono{Droid Sans Mono}
%\newfontfamily\dejavumono{DejaVu Sans Mono}

\lstdefinelanguage{nesc}
  {morekeywords={components, configuration, event, generic, implementation, includes, interface, module,new, norace, post, provides, signal, task, uses,nx\_struct, nx\_union,command,uint16\_t,uint8\_t,uint32\_t,as,void},sensitive=false,morecomment=[l]{//},morecomment=[s]{/*}{*/},morestring=[b]",}

\lstset{basicstyle=\dejavumono\scriptsize,keywordstyle=\color{blue},commentstyle=\color{green},stringstyle=\color{red},tabsize=2,frameround=ffff,escapeinside=``,lineskip=1pt,framerule=0.5pt,xleftmargin=20pt,xrightmargin=10pt,language=nesc,frame=single,numbers=left,framexleftmargin=6mm}
%\lstset{basicstyle=\droidmono\footnotesize,tabsize=4,frameround=ffff,escapeinside=``,lineskip=1pt,framerule=0.5pt,xleftmargin=20pt,xrightmargin=10pt,language=nesc,frame=tb,captionpos=b,abovecaptionskip=10pt,numbers=left, framexleftmargin=5mm}

\renewcommand\arraystretch{1.25}


%%%%%%%%%%%%%%%%%%%%%%%%%%%%%%%%%%%%%%%%%%%%%%%%%%%%%%%%%%%
%
% 正文部分
%
%%%%%%%%%%%%%%%%%%%%%%%%%%%%%%%%%%%%%%%%%%%%%%%%%%%%%%%%%%%

%--- Preface ------------------------
\frontmatter

% 解决中英文混排的断行问题,会加入间距,但不会影响断行
\sloppy

\pagenumbering{Roman}

%封面
% !Mode:: "TeX:UTF-8"
%%%%%%%%%%%%%%%%%%%%%%%%%%%%%%%%%%%%%%%%%%%%%%%%%%%%%%%%%%%%%%%%%%%%%%%%%
%
%   LaTeX File for Doctor (Master) Thesis of Tsinghua University
%   LaTeX + CJK     清华大学博士(硕士)论文模板
%   Based on Wang Tianshu's Template for XJTU
%   Version: 1.00
%   Last Update: 2003-09-12
%
%%%%%%%%%%%%%%%%%%%%%%%%%%%%%%%%%%%%%%%%%%%%%%%%%%%%%%%%%%%%%%%%%%%%%%%%%
%   Copyright 2002-2003  by  Lei Wang (BaconChina)       (bcpub@sina.com)
%%%%%%%%%%%%%%%%%%%%%%%%%%%%%%%%%%%%%%%%%%%%%%%%%%%%%%%%%%%%%%%%%%%%%%%%%

%%%%%%%%%%%%%%%%%%%%%%%%%%%%%%%%%%%%
% 封一
%%%%%%%%%%%%%%%%%%%%%%%%%%%%%%%%%%%%

\begin{titlepage}
\begin{center}
%\begin{minipage}[c]{2.64cm}
%\centering
%\resizebox{!}{0.9cm}{%
%\parbox{0.54cm}{\input{logo}}
%}
%\end{minipage}
% \hskip 0.8cm
%\begin{minipage}[c]{8cm}
%\fontsize{33}{33}\nwpulogo 西北工业大学
\sanhao\song {西~~北~~工~~业~~大~~学}
%\end{minipage}
\\
\yihao\song {硕~~士~~学~~位~~论~~文}
\vskip 0.2cm
\sihao\song {(学位研究生)}

\vskip 4cm
\erhao
题目:\underline{\hspace{2.5em}基于CD-Phone和CTC的\hspace{2.5em}}
\hspace*{3em}{\underline{\hspace{0.5em}语音识别技术研究\hspace{0.5em}}}

\vskip 5cm
\sanhao\song 作\hspace{2em}者:\coverunderline[5cm]{张彬彬}
\vskip 0.1cm
\sanhao\song 学科专业:\coverunderline[5cm]{计算机应用技术}
\vskip 0.1cm
\sanhao\song 指导老师:\coverunderline[5cm]{谢磊~~教授}
\vskip 2cm
\sanhao\song{2017~~年~~~~3~~~~月}
\vfill
\end{center}
\end{titlepage}
\clearpage



% XX: english cover
\newpage
\thispagestyle{empty}
\begin{center}
\vspace*{22pt}
%\erhao % 22pt
\fontsize{22pt}{26pt}\selectfont
\textbf{Title: CD-Phone and CTC based Speech Recognition}
\vskip 2cm
%\xiaosan % 15pt
\fontsize{15pt}{17pt}\selectfont
\textbf{By\\Zhang Binbin}
\vskip 0.5cm
\textbf{Under the Supervision of Professor\\Xie Lei}

\vskip 2cm
%\sanhao % 16pt
\fontsize{16pt}{18pt}\selectfont
A Dissertation Submitted to\\ Northwestern Polytechnical University
\vskip 0.5cm
In partial fulfillment of the requirement\\ For the degree of\\
Master of Computer Application Technology

\vskip 3cm
%\xiaosan % 15pt
\fontsize{15pt}{17pt}\selectfont
Xi'an P.R. China\\March 2017
\vfill
\end{center}

\song \normalsize



%授权
%% !Mode:: "TeX:UTF-8"

\newpage
\thispagestyle{empty}
\markboth{西北工业大学学位论文知识产权说明书}{西北工业大学学位论文原创性声明}

\vspace*{0.1cm}
%\newcommand{\subchapterstyle}%
%  {\CJKfamily{hei}\rmfamily\bfseries\fontsize{14pt}{14pt}\selectfont}

\newcommand{\subchapterstyle}{\song\bfseries\fontsize{14pt}{14pt}}
\newcommand{\twospace}{~~~~~~~~}

%\phantomsection
%\addcontentsline{toc}{chapter}{\hei\xiaosi 西北工业大学学位论文知识产权说明书}
%\addcontentsline{toe}{chapter}{\bfseries\xiaosi Statement of Copyright}
\begin{center}{\subchapterstyle 西北工业大学 \\ \vspace*{0.1cm} 学位论文知识产权说明书}\end{center}

    {\fontsize{10.5pt}{13pt}\selectfont \twospace 本人完全了解学校有关保护知识产权的规定,即:研究生在校攻读学位期间论文工作的知识产权单位属于西北工业大学。
    学校有权保留并向国家有关部门或机构送交论文的复印件和电子版。本人允许论文被查阅和借阅。
    学校可以将本学位论文的全部或部分内容编入有关数据库进行检索,可以采用影印、缩印或扫描等复制手段保存和汇编本学位论文。
    同时本人保证,毕业后结合学位论文研究课题再撰写的文章一律注明作者单位为西北工业大学。


    \twospace 保密论文待解密后适用本声明。


    \twospace 学位论文作者签名:\underline{\quad\quad\quad\quad\quad\quad\quad} \hspace{8em}指导教师签名:\underline{\quad\quad\quad\quad\quad\quad\quad}

    \hspace{5em} \quad 年 \quad 月 \quad 日 \hspace{17em} 年 \quad 月 \quad 日 \quad}





\vspace{1\baselineskip}
---------------------------------------------------------------------------------------------------------
\vspace{1\baselineskip}

%\phantomsection
%\addcontentsline{toc}{chapter}{\hei 西北工业大学学位论文原创性声明}
%\addcontentsline{toe}{chapter}{\bfseries\xiaosi Letter of Authorization}
\begin{center}{\subchapterstyle 西北工业大学 \\ \vspace*{0.1cm}学位论文原创性声明}\end{center}

 {\fontsize{10.5pt}{13pt}\selectfont \twospace 秉承学校严谨的学风和优良的科学道德,本人郑重声明:所呈交的学位论文,是本人在导师的指导下进行研究工作所取得的成果。
 尽我所知,除文中已经注明引用的内容和致谢的地方外,本论文不包含任何其他个人或集体已经公开发表或撰写过的研究成果,
 不包含本人或其他已申请学位或其他用途使用过的成果。对本文的研究做出重要贡献的个人和集体,均已在文中以明确方式表明。

\twospace 本人学位论文与资料若有不实,愿意承担一切相关的法律责任。

\vspace{0.5\baselineskip}

\hspace{14em} 学位论文作者签名:\underline{\quad\quad\quad\quad\quad\quad\quad}

\hspace{18em} \quad 年 \quad 月 \quad 日}


% \BiAppendixChapter{西北工业大学学位论文原创性声明及使用授权说明}{Statement of copyright and Letter of authorization}
% \vspace{\baselineskip}
% \begin{center}\hei\xiaosan{学位论文原创性声明}\end{center}
% \vspace{1em}

% 本人郑重声明:此处所提交的学位论文《\chinesethesistitle》,是本人在导师指导下,在哈尔滨工业大学攻读学位期间独立进行研究工作所取得的成果。据本人所知,论文中除已注明部分外不包含他人已发表或撰写过的研究成果。对本文的研究工作做出重要贡献的个人和集体,均已在文中以明确方式注明。本声明的法律结果将完全由本人承担。

% \vspace{\baselineskip}
% \hspace{6em}作者签名:\hfill 日期:\hspace{2.5em}年\hspace{1.5em}月\hspace{1.5em}日

% \vspace{3\baselineskip}
% \begin{center}\hei\xiaosan{学位论文使用授权说明}\end{center}
% \vspace{1em}

% 本人完全了解哈尔滨工业大学关于保存、使用学位论文的规定,即:

% (1)已获学位的研究生必须按学校规定提交学位论文;(2)学校可以采用影印、缩印或其他复制手段保存研究生上交的学位论文;(3)为教学和科研目的,学校可以将学位论文作为资料在图书馆及校园网上提供目录检索与阅览服务;(4)根据相关要求,向国家图书馆报送学位论文。

% 保密论文在解密后遵守此规定。
% \vspace{\baselineskip}

% 本人保证遵守上述规定。

% \vspace{2\baselineskip}
% \hspace{6em}作者签名:\hfill 日期:\hspace{2.5em}年\hspace{1.5em}月\hspace{1.5em}日

% \vspace{2\baselineskip}
% \hspace{6em}导师签名:\hfill 日期:\hspace{2.5em}年\hspace{1.5em}月\hspace{1.5em}日


\setcounter{page}{1}

%中文摘要
%%%%%%%%%%%%%%%%%%%%%%%%%%%%%%%%%%%%%%%%%%%%%%%%%%%%%%%%%%%%%%%%%%%%%%%%%
%
%   LaTeX File for Doctor (Master) Thesis of Tsinghua University
%   LaTeX + CJK     清华大学博士\KH{硕士}论文模板
%   Based on Wang Tianshu's Template for XJTU
%   Version: 1.00
%   Last Update: 2003-09-12
%
%%%%%%%%%%%%%%%%%%%%%%%%%%%%%%%%%%%%%%%%%%%%%%%%%%%%%%%%%%%%%%%%%%%%%%%%%
%   Copyright 2002-2003  by  Lei Wang (BaconChina)       (bcpub@sina.com)
%%%%%%%%%%%%%%%%%%%%%%%%%%%%%%%%%%%%%%%%%%%%%%%%%%%%%%%%%%%%%%%%%%%%%%%%%


%%%%%%%%%%%%%%%%%%%%%%%%%%%%%%%%%%%%%%%%%%%%%%%%%%%%%%%%%%%%%%%%%%%%%%%%%
%
%   LaTeX File for phd thesis of xi'an Jiao Tong University
%
%%%%%%%%%%%%%%%%%%%%%%%%%%%%%%%%%%%%%%%%%%%%%%%%%%%%%%%%%%%%%%%%%%%%%%%%%
%   Copyright 2002  by  Wang Tianshu    (tswang@asia.com)
%%%%%%%%%%%%%%%%%%%%%%%%%%%%%%%%%%%%%%%%%%%%%%%%%%%%%%%%%%%%%%%%%%%%%%%%%
\renewcommand{\baselinestretch}{1.5}
\fontsize{12pt}{13pt}\selectfont

\chapter{摘~~~~要}
\vspace{2em}
\markboth{摘~~要}{摘~~要}

随着我国经济水平的提高,人们消费意识和水平的提升,旅游业在国民经济中占有越来越重要的地位。

\vspace{1em}
{\hei 关键词:} \quad 关键词,关键词2



%英文摘要
%%%%%%%%%%%%%%%%%%%%%%%%%%%%%%%%%%%%%%%%%%%%%%%%%%%%%%%%%%%%%%%%%%%%%%%%%
%
%   LaTeX File for Doctor (Master) Thesis of Tsinghua University
%   LaTeX + CJK     清华大学博士(硕士)论文模板
%   Based on Wang Tianshu's Template for XJTU
%   Version: 1.00
%   Last Update: 2003-09-12
%
%%%%%%%%%%%%%%%%%%%%%%%%%%%%%%%%%%%%%%%%%%%%%%%%%%%%%%%%%%%%%%%%%%%%%%%%%
%   Copyright 2002-2003  by  Lei Wang (BaconChina)       (bcpub@sina.com)
%%%%%%%%%%%%%%%%%%%%%%%%%%%%%%%%%%%%%%%%%%%%%%%%%%%%%%%%%%%%%%%%%%%%%%%%%

%%%%%%%%%%%%%%%%%%%%%%%%%%%%%%%%%%%%%%%%%%%%%%%%%%%%%%%%%%%%%%%%%%%%%%%%%
%
%   LaTeX File for xi'an Jiao Tong University
%
%%%%%%%%%%%%%%%%%%%%%%%%%%%%%%%%%%%%%%%%%%%%%%%%%%%%%%%%%%%%%%%%%%%%%%%%%
%   Copyright 2001  by  Wang Tianshu    (tswang@asia.com)
%%%%%%%%%%%%%%%%%%%%%%%%%%%%%%%%%%%%%%%%%%%%%%%%%%%%%%%%%%%%%%%%%%%%%%%%%
\renewcommand{\baselinestretch}{1.5}
\fontsize{12pt}{13pt}\selectfont

\chapter[\textbf{ABSTRACT}]{\textrm{\textbf{Abstract}}}
\vspace{2em}
\markboth{Abstract}{Abstract}
\noindent
Run  the  TeX  typesetter on file, usually creating file.dvi.
If the file argument has no extension, ".tex" will be appended to it.
Instead of a filename, a set of TeX commands can be given, the first of which must start with a backslash.
With a \&format argument TeX uses a different set of  precompiled  commands,
contained in format.fmt; it is usually better to use the -fmt format option instead.

\vspace{1em}
\noindent {\textbf{KEY WORDS:}} \quad Internet of Things, TinyOS, Sensor Node, IPv6

%目录
\renewcommand{\baselinestretch}{1.5}
%\fontsize{12pt}{12pt}\selectfont
\xiaosi
\tableofcontents
\addcontentsline{toc}{chapter}{目录}

%符号对照表
%\include{preface/denotation}

\mainmatter

\renewcommand{\baselinestretch}{1.5}

% 对应于小四的标准字号是 12pt
% 可以在正文中用此命令修改所需要字体的的大小
%\fontsize{12pt}{13pt}\selectfont
\xiaosi\song


%--- body --------------------------

%正文章节

% !Mode:: "TeX:UTF-8"
\chapter{绪论}

语音是人类最自然最便捷的交互方式,语音识别(Automatic Speech Recognition)则是语音交互中最为核心的技术,它将用户语音输入识别为文字,从而让计算机“理解”用户需求,
声学建模又是语音识别中的核心技术。本文研究基于CD-Phone\ucite{senior2015context, sak2015fast, sak2015learning}(Context Dependent Phone)和CTC\ucite{graves2012neural, graves2006connectionist}(Connectionist  Temporal Classification)的声学建模技术。
本章首先介绍了本文的研究背景及意义;然后介绍了语音识别技术的研究现状;接着介绍了本文的主要工作;最后给出本文的章节安排。


\section{研究背景及意义}

随着移动互联网和智能可穿戴设备的蓬勃发展,作为人类最自然最便捷的交互方式的语音在人机交互中扮演越来越重要的角色。对于智能移动终端和可穿戴设备,由于其便携性高,体积小,
用户的交互和输入方式及其有限。智能手机中人们尚可以通过屏幕进行输入和交互,但对于智能手表,智能手环,智能眼镜等的可穿戴设备,其屏幕很小或者不存在屏幕,语音甚至是唯一的交互方式。
同样在工业、家电,通信,汽车电子,医疗,智能家居等行业中,语音也扮演了越来越重要的角色。语音识别技术则是语音交互中最核心最复杂的技术之一。

近年以来,作为移动互联网的重要入口,语音发挥着越来越重要的作用。互联网巨头纷纷在语音识别领域纷纷投入巨资,并推出一系列产品。国外苹果公司在其众多移动终端设备(iphone,ipad,apple watch)中推出个人语音助手Siri;微软在其Windows设备中推出个人语音助理Contana;Google在Google Search, Android设备中推出了Google Now, 语音科技公司Nuance推出智能听写助手Dragon Assistant。而国内竞争更是日益激烈,百度、讯飞、阿里巴巴、搜狗等纷纷推出其语音产品,如讯飞语音助手,百度语音搜索和输入法,出门问问在其智能手表ticwear同样搭载了语音交互系统。
语音识别在我们的日常生活中扮演了越来越重要的角色。

声学建模是语音识别系统的核心的技术之一。传统的语音识别使用基于HMM-GMM\ucite{rabiner1993fundamentals}(Hidden Markov Model-Gaussion Mixture Model)的方法进行声学建模,GMM方法在本质上是一种浅层模型,其建模能力有限。2010年以后,深度神经网络DNN\ucite{hinton2012deep, dahl2012context}(Deep Neural Network)的深层模型的应用,大幅度提高了声学建模的精度,使得语音识别系统的性能出现质的飞跃。近来以RNN(Recurrent Neural Network)和LSTM(Long Short-Term Memory)
\ucite{hochreiter1997long, graves2012supervised, sak2014long, sak2014long_lvsr}
为代表的循环神经网络的应用,使得神经网络拥有了记忆功能和时序建模能力,进一步提高了神经网络声学建模的精度。

传统的声学建模以HMM状态CD-State(Context Dependency State)作为基本建模单元,HMM-GMM系统中使用GMM拟合状态的概率密度分布;
HMM-DNN(Hidden Markov Model-Deep Neural Network)系统中使用DNN对状态进行建模,通过状态间的跳转来描述不同音素(Phone)的发音过程。
基于RNN和LSTM的神经网络通过引入Recurrent层,使得神经网络具有了时序建模能力和记忆功能。
时序建模和记忆功能的引入,通过RNN直接对音素发音的过程进行建模,使得以更大的单元(Phone)作为基础建模单元成为可能。
2015年,Google首次\ucite{senior2015context, sak2015fast, sak2015learning}提出以循环神经网络作为基本网络结构,以Phone和CD-Phone(Context Dependent Phone)为建模单元的声学建模技术,
并且超越了传统的基于CD-State的语音识别系统。

传统的分类任务的神经网络使用交叉熵CE(Cross Entropy)作为基本的训练准则。
以交叉熵为优化准则,使用神经网络建模的声学模型需要依赖传统的HMM-GMM系统产生训练所需的状态对齐和决策树。
基于CTC的优化准则通过引入Blank,在整个序列上进行迭代优化,使得深度神经网络具有了end-to-end的建模能力。
同时,基于CTC优化准则的分类预测具有尖峰(Peak)预测特性,可以使分类更加精准鲁棒,因而可以得到声学建模精度的进一步提升。
2015年,Google、百度的研究\ucite{senior2015context, sak2015fast, sak2015learning, hannun2014deep, amodei2015deep}表明,
CTC不仅可以进一步提升声学建模的精度;而且由于尖峰预测的特性,基于CTC的语音识别解码器具有非常快的解码速度。

综上所述,基于CD-Phone和CTC的声学建模技术有如下优点:
第一,基于CD-Phone和CTC的识别系统,均可进一步提升了声学建模的精度;
第二,CTC的end-to-end建模能力无需依赖传统HMM-GMM系统,可以大大简化语音识别系统的流程;
第三,更大的单元CD-Phone的引入减小了解码网络,基于CTC的语音识别解码器具有非常快的解码速度,两者的共同作用,
可以大幅度提升语音识别系统的响应速度和吞吐量。
因此,基于CD-Phone和CTC的语音识别技术研究具有非常重要的意义。

\section{研究现状}

语音识别技术起步于上世纪50 年代,1960 年英国Denes等人研究成功了第一个计算机语音识别系统。
在70 年代时,使用标准模板匹配的方法,语音识别小词汇量、孤立词的识别上取得重大进展。
进入80 年代以后,语音识别研究重点逐渐转向大词汇量、非特定人的连续语音识别,
研究方法也由传统的标准模板匹配转向基于统计的隐马尔科夫(HMM)模型,HMM 的提出和应用是语音识别技术的重大突破和转折。
HMM因其对音素的有效建模便一直作为语音识别系统中的核心方法之一沿用至今。

经过几十年的发展,HMM-GMM的语音识别系统已经取得了长足进展,HMM-GMM应用于语音识别的理论和实践逐渐完善,并诞生了一系列语音识别的新技术。
如基于最大后验概率准则估计(Maximum A-Posteriori Estimation, MAP Estimation)和
基于最大似然线性回归(Maximum Likelihood Linear Regression, MLLR)自适应技术;
基于状态态绑定的决策树技术;基于最大互信息MMI(Maximum Mutual Information)和MCE(Minimum Classification Error)等准则的区分度训练技术等等。
此时,基于HMM-GMM的传统语音识别系统已经能够在特定任务上取得理想的识别精度。
移动互联网的到来,为语音识别技术带来新的挑战。
在移动互联网和智能可穿戴设备应用中,由于不同的环境噪声、不同信道、口音、录音设备等差异性,对语音识别的精度和性能提出更高的要求。

20世纪80年代末至90年代初,研究人员已经探索ANN(Artificial Neural Network)和HMM在语音识别中的应用,但由于当时硬件计算能力、数据和训练方法的限制,
研究者基本均使用两个隐层以下的神经网络,表达能力有限,而深层的神经网络又难以训练,
所以当时基于ANN-HMM的语音识别系统很难超越经典的HMM-GMM方法,并未获得成功。

2006年, Geoffrey Hinton提出了DBN(Deep Belief Networks),
在训练之前,DBN使用无监督的受限制的玻尔兹曼机(Unsupervised Restricted Boltzmann Machine)对DNN的进行逐层预训练,
从而解决深度神经网络的训练收敛性问题。

2010年以来,深度神经网络DNN在语音识别中的成功应用\ucite{dahl2012context, dahl2011large},使得语音识别系统的性能出现质的飞跃。
相对于传统HMM-GMM语音识别系统,基于深度神经网络
的语音识别系统HMM-DNN能够普遍使得语音识别错误率WER(Word Error Rate)得到20\%~30\%的下降,大幅度的提高了声学建模的精度。
目前,基于HMM-DNN的语音识别系统日益成熟,并成功在主流的语音识别商业产品中使用。

DNN

CNN

LSTM

CTC

CD-State CD-Phone

神经网络的并行训练。



近年以来,以RNN(Recurrent Neural Network)和LSTM(Long Short Term Memory)为代表的循环神经网络的应用,进一步提高了神经网络声学建模的精度。循环神经具有记忆单元,从而可以对长时序进行建模,进一步可以对更大的建模单元进行建模。研究学者纷纷探索和挖掘循环神经网络更多潜在的能力,以CD-Phone(Context Dependent Phone)新建模单元和CTC(Connectionist  Temporal Classification)新建模技术日益得到广泛关注。


\section{本文工作}

本文所作的主要工作如下:
\begin{enumerate}
\item 设计数据采集节点上的传感器硬件连接图,完成几种传感器驱动在TinyOS操作系统上的实现。
\item 开发基于Web服务的支持IPv6协议栈的传感器节点,其可以将传感数据汇聚到网关模块。
\item 设计开发支持4G移动网络的远程网关程序,其可以收集传感数据并通过4G网络将数据上报至数据展示子系统。
\item 针对采集的传感数据,进行数据库的设计并且开发展示数据的Web网站。
\end{enumerate}

% 1.4
\section{章节安排}

第一章即本章为绪论,总述课题的来源与研究背景与意义,
确定论文主体工作围绕基于Web服务的支持IPv6的传感节点展开。

第二章提出了项目的总体需求,并提出了系统的系列功能需求,其中传感器数据的采集与展示为基础功能。
最后结合总体需求和功能需求提出了系统的总体设计,包括各个子系统结构与功能,以及协议栈与其数据格式表示。

第三章根据项目的传感数据进行传感器的选型与软硬件设计,
包括各种传感器的引脚的连接图,以及针对具体传感器实现其驱动程序。

第四章首先介绍传感节点所依赖的IPv6协议栈,分析了教研室移植开发的6LoWPAN协议栈TinyV6,
包括无线传感器节点之间的适配层与汇聚节点与上位机之间的PPP协议等链路层,
以及网络层的路由与自动配置与传输层、套接字接口等,结合协议栈特点实现基于Web服务的IPv6的监测节点部分。

第五章根据实际部署情况,提出利用4G功能的网关来支持远程部署,并且进一步实现了基于Web服务的网关模块。

第六章描述了整个监测系统的展示子系统,包括数据库的设计, 网站的交互设计与数据展示。

第七章总结了全文的工作,结合当前监测系统的特点与不足,提出了系统的进一步设计改进的方案。
 %绪论

% 结论
%\include{body/summary}

%参考文献
\wuhao

\bibliographystyle{unsrt}

\ifpdf \phantomsection \fi

\addcontentsline{toc}{chapter}{参考文献}

%\addtolength{\itemsep}{-0.8 em} % 缩小参考文献间的垂直间距, 在bibtex下无效
\addtolength{\bibsep}{-1ex}
\bibliography{ref/ref}

% 致谢
\xiaosi\song
% !Mode:: "TeX:UTF-8"

\BiAppendixChapter{致\quad 谢}{Acknowledgements}

\vspace{2ex}
岁月如白驹过隙,硕士转瞬即尽,彷徨过,迷惘过,有过辛酸,有过汗水,也有些许成就。
在技术上,三年的教研室的科研生涯使我的科研和工程能力得到进一步提高,能够独当一面;通过外出实习,个人视野更加的广阔,
对个人研究方向在工业界的大规模应用有了更清晰的认识,更好的把握时代的前沿科技。
在生活上,自信而自律,文艺而又不失洒脱,能够在生活和工作上找到自己的平衡点。


首先,特别感谢我的导师谢磊教授。谢老师视野开阔,工作勤奋,对新的技术和事物有着很强的把握;
孜孜不已,诲人不倦,以身做则,才有如今“谢家桃李满天下”;他的勤奋和努力始终鼓舞着我。
在科研工作上,谢老师经常与我们共同讨论,提出自己的想法和意见。在实习工作上,谢老师主动努力为我们提供更好的
实习和工作机会。

感谢实验室的各位同学,感谢那些与你们一起科研,一起Happy,一起疯狂的日子。

感谢我的诸位室友,谢谢你们的陪伴与互勉,让我三年的研究生生涯异样的精彩。

当然,从前种种譬如昨日之死,以后种种譬如今日之生。这又是人生的新起点!


\quad\quad\quad\quad\quad\quad\quad\quad\quad\quad\quad\quad\quad\quad\quad\quad\quad\quad\quad\quad\quad\quad\quad\quad\quad\quad\quad\quad\quad\quad 张彬彬

\quad\quad\quad\quad\quad\quad\quad\quad\quad\quad\quad\quad\quad\quad\quad\quad\quad\quad\quad\quad\quad\quad\quad\quad\quad\quad\quad\quad\quad 2017年3月

%衷心感谢导师~XXX~教授对本人的精心指导。他的言传身教将使我终生受益。感谢~XXX~教授,以及实验室全体老师和同窗们的热情帮助和支持!本课题承蒙~XXXX~基金资助,特此致谢。



\clearpage\mbox{}

%\include{appendix/design-conclusion}

%  附录

%\begin{appendix}
%    \renewcommand{\chaptername}{附录\Alph{chapter}}
%   \input{appendix/appendix.tex}
%\end{appendix}

% 发表的文章列表

\xiaosi\song
% !Mode:: "TeX:UTF-8"

\defaultfont
\BiAppendixChapter{攻读\cxuewei 学位期间发表的论文及其他成果} {Papers published in the period of PH.D. education}
\setlength{\parindent}{0em}
\textbf{一、发表的学术论文}
\begin{publist}
  \item {\bf Xiangzeng Zhou}, Lei Xie, Peng Zhang and Yanning Zhang. Online Object Tracking Based on BLSTM-RNN with Contextual-Sequential Labeling[C]. \textcolor{blue}{IEEE Transactions on Multimedia (TMM)} (under review).(\underline{\bf SCI 2区})
\end{publist}
\vspace{1cm}

\textbf{二、博士期间的获奖情况}
\begin{publist}
 \item ICIP 2014 Top\%10 Paper Award.({\bf 第一作者})
 \item 第10届京港国际博士生论坛Best Paper.({\bf 第一作者})
 \item 获2016年西北工业大学大学计算机学院“图灵之星”奖
 \item 获2015年西北工业大学“理光奖学金”
 \item 获2015年西北工业大学“海泰奖学金”
\end{publist}
% \textbf{二、参与的科研项目及获奖情况}
% \begin{publist}
% \item XX. Towards A Queue-Aware ATM: Monitoring and Managing Queues in Front of ATMs, NCR英国国际合作项目.课题编号:XXXX.
% \item XX. 语音内容分析的关键技术研究, 陕西省自然科学基础研究计划.课题编号:XXXX.
% \item XX. 基于DBN协同建模的中文及跨语种语音结构事件检测研究, 国家自然科学基金.课题编号:XXXX.
% \item XX. 机载语音处理技术,	研究所合作项目.课题编号:XXXX.
% \end{publist}
% \vfill
% \hangafter=1\hangindent=2em\noindent

% \setlength{\parindent}{2em}


\clearpage
\end{document}

%%%%%%%%%%%%%%%%%% End of the file  %%%%%%%%%%%%%%%%%%%%%%%%
