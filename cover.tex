% !Mode:: "TeX:UTF-8"

\newcommand{\chinesethesistitle}{基于CD-Phone和CTC的语音识别技术研究} %授权书用,无需断行
\newcommand{\englishthesistitle}{\uppercase{CD-Phone and CTC based Speech Recognition}} %\uppercase作用:将英文标题字母全部大写;
\newcommand{\chinesethesistime}{2017~年~3~月}  %封面底部的日期中文形式
\newcommand{\englishthesistime}{March 2016}    %封面底部的日期英文形式

\ctitle{基于CD-Phone和CTC的语音识别技术研究}  %封面用论文标题,自己可手动断行
\cdegree{\cxueke\cxuewei}
\csubject{计算机应用技术}                 %(~按二级学科填写~)
\caffil{计算机学院} %(在校生填所在系名称,同等学力人员填工作单位)
\cauthor{张彬彬}
\studentnum{2014201751}
\csupervisor{谢磊~教授} %导师名字
%\cassosupervisor{谢磊~教授}%若没有,请屏蔽掉此句。
%\ccosupervisor{联导名}%若没有,请屏蔽掉此句。
\cdate{\chinesethesistime}
\etitle{\englishthesistitle}
\edegree{\exuewei \ of \exueke}
\esubject{Computer Science and Technology}  %英文二级学科名
\eaffil{School of Computer Science}
\eauthor{Zhang Binbin}                   %作者姓名 (英文)
\esupervisor{Prof. Lei Xie}       % 导师姓名 (英文)
%\eassosupervisor{Prof. Assosuper}%若没有,请屏蔽掉此句。
%\ecosupervisor{Prof. Cosuper}%若没有,请屏蔽掉此句。
\edate{\englishthesistime}
\natclassifiedindex{TM301.2}  %国内图书分类号
\internatclassifiedindex{62-5}  %国际图书分类号
\statesecrets{公开} %秘密
\iffalse
  \BiAppendixChapter{摘~~~~要}{}  %使用winedt编辑时文档结构图(toc)中为了显示摘要,故增加此句;
\fi

\cabstract{
语音是人类最为自然、最为便捷的人机交互方式,语音识别则是语音交互技术的核心技术之一。
目前,基于深度学习的语音识别技术取得了巨大成功。
本文研究基于深度学习和传统建模单元CD-State的声学建模,研究基于CD-Phone和CTC的新型声学建模技术,
并进一步研究基于深度学习的声学模型的并行训练。

首先,研究以传统CD-States作为建模单元,分别使用DNN、CNN、LSTM和混合神经网络CLDNN进行声学建模。
在本文试验中,相对于简单的DNN,CNN、LSTM和CLDNN能够分别使得语音识别错误率普遍降低3\% \textasciitilde 5\%、
5\% \textasciitilde 20\%和10\% \textasciitilde 23\%。在CLDNN基础上进一步使用跳帧训练技术,
在保证声学模型精度的前提下,成功将训练速度提升3倍。

然后,研究以CD-Phone和CTC为代表的新型语音识别技术。
文中分别以Phone、CD-Phone、Syllable作为建模单元,并结合CTC准则进行优化,
最终基于Phone和CTC的系统取得了与基于CD-State和CLDNN的系统相当的识别结果。
并且,使用CTC的将识别系统的解码速度提升3倍以上。

最后,本文研究基于深度学习的声学模型的并行训练技术。
文中实现基于BSP、ASGD、EASGD和BMUF四种并行训练算法,并在DNN和CLDNN模型上进行实验比较其性能。
最终,基于BMUF和ASGD的并行算法取可以取得了较优的模型结果。
BMUF算法更为稳定;ASGD加速比更好,但需要更为精细的初始化。
使用4卡、基于BMUF和ASGD的的并行方法能在基本不损失模型精度的情况下,
相对单卡取得3.5倍以上的加速比。

}
\ckeywords{CLDNN, CD-Phone, CTC, 并行训练} %最后一个关键词后面没有标点符号

\eabstract{
Speech is the most natural and convenient way for human-computer interaction, 
and speech recognition is one of the core technology for speech interaction.
Now deep learning based speech recognition get the state of art performance.
In this paper, we study deep learning and classic model unit CD-State based acoustic modeling, 
CD-Phone and CTC based end-to-end acoustic modeling,
and parallel training algorithms for deep learning based speech recognition.

First, we study CD-State based model unit, apply DNN, CNN, LSTM and CLDNN to the acoustic model step by step.
Compared with simple DNN, we get 3\% \textasciitilde 5\%, 5\% \textasciitilde 20\%, 10\% \textasciitilde 23\% 
CER reduction by using CNN, LSTM, CLDNN respectively. Also, we study skip training on CLDNN, 
successfully improve the training speed by 3 times without CER reduction.

Then, we study CD-Phone and CTC based end-to-end speech recognition. 
We use Phone, CD-Phone and Syllable as model unit, combined with CTC optimization criteria, 
get comparable speech recognition result to CD-State and CLDNN based acoustic modeling.
Further, decoding speed for CTC based speech recognition system is more than 3 times faster.

Last, several parallel training algorithms for deep learning based acoustic model are studied. 
we implement 4 parallel training algorithms(BSP, ASGD, EASGD and BMUF) in this paper, 
and apply these parallel training algorithms on DNN and CLDNN based acoustic modeling respectively.
Finally, we find that BMUF and ASGD based parallel training get the best results, 
BMUF is rather stable and converges faster, ASGD is more fast but requires more careful initialization. 
BMUF and ASGD based parallel training achieve more than 3.5 times speedup by using 4 GPU, without loss precision of the acoustic model.
}
%Externally pressurized gas bearing has been widely used in the field of aviation, semiconductor, weave, and measurement apparatus because of its advantage of high accuracy, little friction, low heat distortion, long life-span, and no pollution. In this thesis, based on the domestic and overseas researching.

\ekeywords{CLDNN, CD-Phone, CTC, Parallel Training} %(no punctuation at the end) 英文摘要与中文摘要的内容应一致,在语法、用词上应准确无误

\makecover
%\clearpage 